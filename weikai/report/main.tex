\documentclass{article}
\usepackage[letterpaper]{geometry}
% \usepackage[letterpaper,margin=1in]{geometry}
\usepackage[utf8]{inputenc}

\title{CS6431 Project Report}
\author{Wei-kai Lin, Yue Guo}
% \date{February 2018}

\usepackage{natbib}
% \usepackage{siunitx} % Provides the \SI{}{} and \si{} command for typesetting SI units
\usepackage{graphicx} % Required for the inclusion of images
\usepackage{natbib} % Required to change bibliography style to APA
\usepackage{amsmath} % Required for some math elements 
\usepackage{amsfonts}
\usepackage{amssymb}
\usepackage{amsmath,amsthm,amstext,amssymb,amsfonts,latexsym}
\usepackage{verbatim}
\usepackage{hyperref}
\usepackage{geometry}
% \geometry{left=3cm, right=3cm, top=3.5cm, bottom=3.5cm}
\usepackage{comment}
\usepackage{enumitem}
\usepackage{color}
\usepackage{parskip} % Remove paragraph indentation

\usepackage{pgfplots}
\pgfplotsset{width = 5.9cm, compat=1.5}

\definecolor{blue1}{rgb}{0.4,0.4,0.7}
\definecolor{blue2}{rgb}{0.3,0.3,0.8}
\definecolor{blue3}{rgb}{0.2,0.2,0.9}

\usepackage{algorithm}
\usepackage[noend]{algpseudocode}
\usepackage{algorithmicx}


\newcommand{\yue}[1]{\textcolor{red}{[yue: #1]}}
\newcommand{\weikai}[1]{\textcolor{blue}{[weikai: #1]}}

\begin{document}
\maketitle

\begin{abstract}
    
\end{abstract}


%!TEX root = main.tex

\section{Introduction}

In the scenario of end-to-end messaging,
a receiver may want to report to a trusted third party 
that the sender has sent abused message.
Note that the server typically needs to verify the report is authentic,
i.e., the sender indeed sent such message.
However, as the encryption is end-to-end,
the server could not know the message had the receiver not reported.
Message franking~\cite{grubbs_message_2017,fb_whitepaper} considers such scenario and proposes schemes
that realizes 1) the security of end-to-end authenticate encryption, and 
2) the feature of verifying an authentic report.
For the sake of exploration,
for each message, 
we say that the message is sent from Sender to the receiver 
in a \emph{sending phase},
where both Sender and Receiver are two clients;
then, in a \emph{reporting phase},
Receiver (w.r.t.~the message) may want to report to the trusted Server
that the message is abused.

In this project, we implement message franking
in the open source library {\tt libsignal},
which is an open-source end-to-end encryption protocol.
The implemented message franking scheme is 
Committing Encrypt-then-PRF (CEP),
and we include the pseudocode in Algorithm~\ref{algo:cep}.

\begin{figure}[t]
\begin{algorithm}[H]
\newcommand{\com}{{\rm com}}
\newcommand{\hmac}{{\rm HMAC}}
	\caption{Scheme of Committing Encrypt-then-PRF (CEP).
	In our implementation, 
	the encrypt/decrypt algorithm $(E,D)$ is AES-CTR;
	symmetric key $K$ and initial vector $N$ 
	are derived using HKDF from the shared secret
	of each message in {\tt libsignal};
	$H$ is the header of each message,
	which will be elaborated in Section~\ref{sec:version_verify};
	$M$ and $C$ are plaintext/ciphertext respectively.
	}
	\label{algo:cep}
	\begin{algorithmic}[1]
	\Procedure{CEP-Enc$^N_K$}{$H, M$}
		\State $P \gets E_K(N, 0^\ell || 0^\ell || M)$
		\State $P_0 \gets P[0:\ell], P_1 \gets P[\ell:2\ell], C \gets P[2\ell:]$
		\State $\com \gets \hmac_{P_0}(H || M)$
		\State $T \gets \hmac_{P_1}(\com)$
		\State \Return $(C || T, \com)$
	\EndProcedure

	\Procedure{CEP-Dec$^N_K$}{$H, C||T, \com$}
		\State $P \gets D_K(N, 0^\ell || 0^\ell || C)$
		\State $P_0 \gets P[0:\ell], P_1 \gets P[\ell:2\ell], M \gets P[2\ell:]$
		\State $\com' \gets \hmac_{P_0}(H || M)$
		\State $T' \gets \hmac_{P_1}(\com')$
		\If{$T \neq T'$ or $\com \neq \com'$}
			\State \Return $\bot$
		\EndIf
		\State \Return $(M, P_0)$
	\EndProcedure

	\Procedure{CEP-Verify}{$H, M, K_f, \com$}
		\State $\com' \gets \hmac_{K_f}(H || M)$
		\If{$\com \neq \com'$} \Return $0$ \EndIf 
		\State \Return $1$
	\EndProcedure
	\end{algorithmic}
\end{algorithm}
\end{figure}







\section{Related Work}
\subsection{Facebook}


\section{Implementation of Message Franking}

Adding message franking to an existing end-to-end encrypted
messaging protocol incurs interface change on both client and server side.
The interface of Sender ({\tt encrypt}) doesn't change,
which takes as input a plaintext
although outputs a cipher message in a new format.
However, the interface of Receiver ({\tt decrypt}),
after takes as input a cipher message,
outputs not only a plaintext but also an opening or \emph{proof},
where the proof is needed 
whenever Receiver wants to report the plaintext to Server.
Finally, to verify a pair of plaintext and proof,
in our implementation,
Server \emph{tags and forwards} the cipher message in the sending phase,
and thus it is necessary to use the new format of the cipher message
to efficiently compute the tag.
In the following, we describe such interface and its usage,
and then, some implementation details are shown 
regarding the security and efficiency.
% \weikai{define sending phase and reporting phase}

\subsection{New Interfaces of Library}

We implement Committing Encrypt-then-PRF on {\tt libsignal-protocol-javascript},
the Signal Protocol library for JavaScript.
The following interface works directly in JavaScript,
but it is straightforward to 
implement them on the library for Java or for C
given that it was already implemented using Protocol Buffers\footnote{
See https://developers.google.com/protocol-buffers/.
}.
\weikai{define CEP}


\paragraph{Client-side Interface: Output of {\tt decryptWhisperMessage}.}
In this paragraph, 
we describe the change of return type of {\tt decryptWhisperMessage},
the function that handling decryption.
The interface of session-building and encryption are unchanged.
We defer the usage of the new interface to Section~\ref{sec:usage}.

The function call to decrypt a cipher message works as follows,
where {\tt ciphertext} is the object returned by the encryption,
and {\tt plaintext} is an array of bytes.
{\small
\begin{verbatim}
sessionCipher.decryptWhisperMessage(ciphertext.body, "binary").then(function (plaintext) {
    console.log(plaintext);
});
\end{verbatim}
}
However, with message franking implemented,
we augment the plaintext into a structure.
In the context of reporting an abuse, we call it ``evidence.''
{\small
\begin{verbatim}
// Evidence
{
    header: ArrayBuffer,         // metadata, returned for HMAC verification
    body: ArrayBuffer,           // the original plaintext
    commitKey: ArrayBuffer(32),  // commitKey and commitment, to verify, check
    commitment: ArrayBuffer(32)  // HMAC(commitKey, concat of (header, body)) == commitment
}
\end{verbatim}
}
Note that such returned structure doesn't include a server generated tag,
which depends on how does server verify the {\tt commitment}.
It is defined in the following paragraph.

\paragraph{Server-side Interface: Cipher Message.}
In scheme of CEP, Server needs to ``know'' 
the {\tt commitment} of each cipher message
in order to verify the plaintext.
In the implementation, 
Server computes a tag of {\tt commitment} using its secret key,
and then forwards both cipher message and tag to Receiver.
We modify the format of cipher messages as follows,
where fields {\tt mac}, {\tt commitment} and {\tt tag}
are modified or added for the purpose of message franking.
\begin{table}[h]
{\small
\begin{verbatim}
{
    type: Unit8, 
    body: ArrayBuffer, concatenation of
        version: 1 byte,
        message: serialized Protocol Buffer of
            ephemeralKey: bytes,
            counter: uint32, 
            previousCounter: uint32, 
            ciphertext: bytes
        mac: 32 bytes,                          // 8 bytes in original format
        commitment: 32 bytes                    // new entry
    registrationId: Uint32, 
    tag: ArrayBuffer(32)                        // new entry, write by Server
}
\end{verbatim}
}
\caption{Cipher Message Structure}
\label{tbl:cipher-message}
\end{table}
Upon forwarding a cipher message,
the server shall use its secret key to compute a tag from {\tt commitment},
and then write the tag to the {\tt tag} field.
The following is our sample code.
{\small
\begin{verbatim}
function signMessage(cipher) {
    var com = getCommitment(cipher);
    return calcHMAC(secretKey, com).then(function (mac) {
        cipher.tag = mac;
        return cipher;
    });
}
\end{verbatim}
}
To report a message,
the client of Receiver has to store {\tt tag} in the cipher message,
as well as {\tt Evidence} returned from {\tt decrypt}.
In the following sample code, 
the server accepts directly the structure of {\tt Evidence} and the {\tt tag},
and then verifies both {\tt tag} and the {\tt commitment} in {\tt Evidence}.
Ideally, such server part should be part of the protocol
even though it is not included in client-side library {\tt libsingal-protocol}.
Also note that a client application shall maintain every pair of 
{\tt evidence, tag} rather than storing data inside {\tt libsingal-protocol}. 
{\small
\begin{verbatim}
function reportAbuse(evidence, tag){
    var macInput = new Uint8Array(evidence.header.byteLength + evidence.body.byteLength);
    macInput.set(new Uint8Array(evidence.header));
    macInput.set(new Uint8Array(evidence.body), evidence.header.byteLength);
    return Promise.all([
        verifyMAC(evidence.commitment, secretKey, tag, 32),
        verifyMAC(macInput, evidence.commitKey, evidence.commitment, 32)
    ]);
}
\end{verbatim}
}

\subsection{Usage}
\label{sec:usage}

The procedures to generate keys and to encrypt are 
identical to the original procedures,
and hence we show only decryption.
Compared to the original procedures,
the only difference is that decryption returns a structure of Evidence,
and that the tag of the server is also returned to 
the application.
{\small
\begin{verbatim}
var address = new libsignal.SignalProtocolAddress(sender.identifier, sender.keyId);
var sessionCipher = new libsignal.SessionCipher(rcver.store, address);
return sessionCipher.decryptPreKeyWhisperMessage(cipher.body, "binary").then(function (evidence) {
    return [evidence, cipher.tag];
});
\end{verbatim}
}
To report an abused message, 
it suffices to send {\tt evidence, cipher.tag} to Server.
It is straightforward and omitted here.


% \subsection{Security and Implementation Details}



% !TEX root = ./main.tex

\section{Performance}
In this section, we test the running time of modified implementation of encryption and decryption algorithm on client,
and the signing time of server for different sizes of messages.
By comparison with the running time of encryption and decryption in the original implementation,
we claim that adding the message franking (also called abuse report below) functionality to \texttt{libsignal} library introduces only very small and totally acceptable time overhead. 


Our implementation is based on the open-source project \texttt{libsignal-protocol.js} of version v1.3.0.
The original \texttt{libsignal} library only deals with the logic at client side.
At the beginning, to test the performance in a more practical environment,
we tried to set up the client and server using the implementation of \texttt{Signal}, an open-source messaging application based on \texttt{libsignal} protocol.
However, setting up the user identity in \texttt{Signal} requires SMS message validation,
which is a little bit tedious and irrelevant to our goal.
So finally we choose to evaluate the performance of simulation in a widely-used browser.
As the whole process 
% sender encrypts - server signs - receiver decrypts, 
is simulated in one process in browser rather than different processes in native environment,
the absolute value of running time might not be of that valuable reference.
Therefore we focus on the comparison between the time consumption of original implementation and our modification.



% Testing environment:
\paragraph{Test environment.}
We test our simulation in Google Chrome browser of version 66.0.3359.170,
with embedded JavaScript of version V8 6.6.346.32.
The Chrome browser runs on Windows 10 operating system,
on a personal computer with Intel Core i7-6700 @ 3.40 Hz CPU and 16.0 GB RAM memory.


Although the physical RAM is large enough to handle longer message encryption,
the memory space we can use is restricted by the sandbox environment of Chrome browser.
In our experiment, the limitation of size of plaintext is between $5 \times 10^7 ~ 10^8$ bytes.
For larger plaintext message, the page will throw error.
If we implement the application (client and server) in local environment,
We can implement the encryption for larger messages.
Besides, even in the browser terminal,
if the user wants to send message or file of large size,
we can always extend the library with large message slicing and different symmetric encryption scheme suitable for files.

To evaluate the time overhead,
we choose message size from $1000$ to $10,000,000$ bytes.
For each data point, we run the whole process of two cases (original implementation and our modification with message franking),
each for 50 times.
We measure the time consumption of each encryption, signing and decryption process
and calculate the average.
The running results are shown in Figure~\ref{graph:performance}.

% libsignal-protocol.js, v1.3.0

% Google Chrome 66.0.3359.170,
% JavaScript V8 6.6.346.32
% Windows 10
% CPU: Intel Core i7-6700 @ 3.40 Hz
% RAM: 16.0 GB

\vspace{0.1in}
\begin{center}
\begin{figure}
\begin{tikzpicture}
% \begin{semilogxaxis}[
\begin{axis}[
    % title = ,
    xlabel = {length of message (byte)},
    ylabel = {time consumed (ms)},
    width = \textwidth * 0.7,
    % log basis x = {10},
    % ytick distance = {0.1},
    % ymin = {0},
    % ymax = {0.6},
    % xtick distance = {100000},
    legend style = {draw = none},
    legend pos = north west,
]
\addplot [blue] table {data/with_report_enc.dat};
\addplot [green]table {data/with_report_sign.dat};
\addplot [red] table {data/with_report_dec.dat};

\addplot [dashed, blue] table {data/without_report_enc.dat};
\addplot [dashed, red] table {data/without_report_dec.dat};

\legend{Enc, Sign, Dec, Enc (original), Dec (original)}
\end{axis}
% \end{semilogxaxis}
\end{tikzpicture}

\caption{Simulation time of Signal with report.}
\label{graph:performance}
\end{figure}
\end{center}
\vspace{0.1in}

From the graph, we can see that in both cases (with or without abuse report functionality),
the encryption time is more than twice of decryption time.
The time of calculating commitment on the sender side
and the time of validating the commitment on the receiver side
is around $5\%$ of the original time consumption.
Adding the abuse report functionality won't significantly impact the performance. 

In addition to time overhead,
adding abuse report functionality also introduces constant space overhead for each message,
as the receiver client now need to store a commitment $C_2$ with each plaintext message.


% !TEX root = ./main.tex

\section{Discussion}
In this section, we discuss two of the major issues we encountered in the process of implementation.
One is whether the participation of server in each message sending is necessary or not,
the other one is which meta data the message authentication process should check.
We explain the existence of these two problems
and propose some preliminary ideas.

% \subsection{API Changing}

\subsection{Server-in-the-Middle is Necessary}
Many of the existing message franking schemes,
including Facebook implementation and our work,
requires that each message from the sender to the receiver be relayed by the server in the middle.
It seems a little bit unsatisfactory from the privacy perspective,
as the server can immediately know when and to whom the sender wants to talk.
Even though the server cannot decrypt the ciphertext,
it still leaks private information and might lead to higher vulnerability to side channel attacks.
So we want to ask here: \emph{can we get rid of the server between the sender and the receiver while having the good abuse report functionality?}

It looks like just a choice of design.
Message relaying through the server is not really necessary to achieve the abuse report functionality.
For example, each pair of user can generate public-secret key pairs, exchange own public keys,
and sign the ciphertext with undeniable signature using the corresponding secret key when sending messages to each other.
The server only stores the public keys of all users.
If one user wants to accuse someone of abusing, it can just send the signed message to the server,
and server can verify if the message is sent by the accused user.
% \yue{Check!}

However, all designs of this kind has an obvious drawback:
the server does not know any more information about the message than any other third-party,
all of the evidence are in the receiver's control. 
If the receiver can prove to the server that a message is indeed sent by the sender,
it can also prove to any other untrusted ones.
It causes privacy concern on the sender side, 
as the sender might not want to disclose to any third party that it has signed on some message.
Actually, passing the message through the server
is just choosing server as a trusted third party,
as the receiver can maliciously disclose any message to the server and prove that the message is indeed sent from the sender,
even if the message is not an abuse message.


Take all this into consideration, we finally choose our current implementation:
the server signs and relays the commitment of ciphertext.
The only extra thing the server needs to store is its own secret key.
The idea is: if I want abuse reporting system, I need to trust some third party;
if I have to trust some party, I choose to trust the server.

\subsection{Version Number Verification}
\label{sec:version_verify}
Message authentication code (MAC) is an important tool to verify the integrity of the message and the meta information.
In \texttt{libsignal}, the MAC is a component of cipher message as shown in Table~\ref{tbl:cipher-message}.
We carefully investigate the original implementation of libsignal,
find that it MACs the whole body but without the \texttt{type} field in the header of the message.

The \texttt{type} field indicates the type of the current message:
\begin{itemize}
\item $\texttt{type} = 3$: PreKeyMessage, the initial message of a new session
\item $\texttt{type} = 1$: the following message in a session that has already been built 
\end{itemize}
These two categories of messages are wrapped in different formats.
The PreKeyMessage has an additional header including more public identities of the sender,
e.g. signed public key.
We argue that the \texttt{type} field should also be MAC-ed,
otherwise there will be a security hole here.
If the \texttt{type} is not included in the integrity check,
an adversary can perform an man-in-the-middle attack by changing the \texttt{type} value.
For example, changing \texttt{type} from 1 to 3 will force the receiver to decrypt a ordinary message as a PreKeyMessage.
Due to the time limit, we are following the original implementation of \texttt{libsignal} now.
We plan to include the type information into the MAC in the next step.


% !TEX root = ./main.tex

\section{Conclusion}
% \subsection{Conclusion}
To conclude,
in this work, we plug the message franking functionality to \texttt{libsignal} library
based on \yue{cite grubb's paper}.
We describe the high level idea and important details in our implementation,
and also give performance analysis. 
The time overhead and space overhead are both in an acceptable level \yue{change narratives},
indicating that we don't need to sacrifies too much performance to get the message franking scheme.
The issues emerging in our implementation are also discussed and
we propose our opinion on how to handle them.

For the next step,
we are planning to 






\bibliographystyle{plain}
\bibliography{references}
\end{document}
