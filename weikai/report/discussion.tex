% !TEX root = ./main.tex

\section{Discussion}
In this section, we discuss two of the major issues we encountered in the process of implementation.
One is whether the participation of server in each message sending is necessary or not,
the other one is which meta data the message authentication process should check.
For these two problems, we state the 
\yue{difficulty of the problem}
and propose some preliminary ideas.

% \subsection{API Changing}

\subsection{Server-in-the-Middle is Necessary}
Many of the existing message franking schemes,
including Facebook implementation and our work,
requires that each message from the sender to the receiver be relayed by the server in the middle.
It seems a little bit unsatisfactory from the privacy perspective,
as the server can immediately know when and to whom the sender wants to talk.
Even though the server cannot decrypt the ciphertext,
it still leaks private information and might lead to higher vulnerability to side channel attacks.
So we want to ask here: \emph{can we get rid of the server between the sender and the receiver while having the good abuse report functionality?}

It looks like just a choice of design.
Message relaying at the server not really necessary to achieve the abuse report functionality.
For example, each pair of user can generate public-secret key pairs, exchange own public keys,
and sign the ciphertext with undeniable signature using the corresponding secret key when sending messages to each other.
The server only stores the public keys of all users.
If one user wants to accuse someone of abusing, it can just send the signed message to the server,
and server can verify if the message is sent by the accused user.
\yue{Check!}

However, all designs of this kind has an obvious drawback:
the server does not know any more information about the message than any other third-party,
all of the evidence are in the receiver's control. 
If the receiver can prove to the server that a message is indeed sent by the sender,
it can also prove to any other untrusted ones.
It causes privacy concern on the sender side, 
as the sender might not want to disclose to any third party that it has signed on some message.
Actually, passing the message through the server
is just choosing server as a trusted third party,
as the receiver can maliciously disclose any message to the server and prove that the message is indeed sent from the sender,
even if the message is not an abuse message.


Take all this into consideration, we finally choose our current implementation:
the server signs and relays the commitment of ciphertext.
The only extra thing the server needs to store is its own secret key.
The idea is: if I want abuse reporting system, I need to trust some third party;
if I have to trust some party, I choose to trust the server.

\subsection{Version Number Verification}
\label{sec:version_verify}
Message authentication code (MAC) is an important tool to verify the integrety of the message and the meta information.
In \texttt{libsignal}, the MAC is a component of cipher message as shown in Table~\ref{tbl:cipher-message}.
We carefully investigate the original implementation of libsignal,
find that it MACs the whole body but without the \texttt{type} field in the header of the message.

The \texttt{type} field indicates the type of the current message:
\begin{itemize}
\item $\texttt{type} = 3$: PreKeyMessage, the initial message of a new session
\item $\texttt{type} = 1$: the following message in a session that has already been built 
\end{itemize}
These two categories of messages are wrapped in different formats.
The PreKeyMessage has an additional header including more public identities of the sender,
e.g. signed public key.
We argue that the \texttt{type} field should also be MAC-ed,
otherwise there will be a security hole here.
If the \texttt{type} is not included in the integrety check,
an adversary can perform an man-in-the-middle attack by changing the \texttt{type} value.
For example, changing \texttt{type} from 1 to 3 will force the receiver to decrypt a ordinary message as a PreKeyMessage.
Due to the time limit, we are following the original implementation of \texttt{libsignal} now.
We plan to include the type information into the MAC in the next step.

